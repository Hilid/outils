\documentclass[a4paper]{article}



    \usepackage[colorinlistoftodos]{todonotes}
 
	\usepackage[utf8]{inputenc}
	\usepackage[T1]{fontenc}
    \usepackage[frenchb]{babel}
    \usepackage{textcomp} 
	\usepackage[top=3cm,left=3cm,right=3cm,bottom=2cm]{geometry}
    \usepackage{lmodern}
    \usepackage{sectsty}
    \usepackage{nicefrac}
	\usepackage{graphicx}
    \usepackage{lastpage}
    \usepackage{fancyhdr}
    \usepackage{amsmath}
    \usepackage{amssymb}
    \usepackage{amsfonts}
    \usepackage{capt-of}
    \usepackage{caption}
    \usepackage{tikz}
    \usepackage{multirow}
    
    \usepackage{fancyvrb} % pour forcer les verbatim sur une seule page
    \usepackage{url}
    
    \newcommand\matlab{MATLab\textsuperscript{\textregistered}}




\title{Compte rendu de TP: Vibrations des milieux continues }
%\subtitle{Basis of room acoustics: TP1}

\author{Thomas Lechat \& Arthur Guibard}

\begin{document}
\maketitle

\section{Introduction}
Page template latex, avec les trucs utiles.

\section{trucs utiles}

\subsection{listes}
\begin{itemize}
	\item $x$ est la différence de hauteur entre les 2 capteurs de pression $i$ ;
    \item $V_i$ est la vitesse au point $i$;
    \item $\rho$ est la masse volumique du fluide à l'étude ;
\end{itemize}

\subsection{figures}
%\begin{figure}[!h]	\centering{\includegraphics[scale=0.5]{img_TP_perte/section_droite1.png}}
%    \caption{\label{perte:sec1} Perte de charge en fonction de la vitesse d'écoulement au carré dans le tuyau foncé}
%\end{figure}


\section{Conclusion}
Enjoy!

\end{document}